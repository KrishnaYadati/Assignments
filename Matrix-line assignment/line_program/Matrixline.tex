\def\mytitle{MATRICES USING PYTHON}
\def\myauthor{YADATI KRISHNA}
\def\contact{yadati.krishna@gmail.com}
\def\mymodule{Future Wireless Communication (FWC)}
\documentclass[10pt, a4paper]{article}
\usepackage[a4paper,outer=1.5cm,inner=1.5cm,top=1.75cm,bottom=1.5cm]{geometry}
\twocolumn
\usepackage{graphicx}
\graphicspath{{./images/}}
\usepackage[colorlinks,linkcolor={black},citecolor={blue!80!black},urlcolor={blue!80!black}]{hyperref}
\usepackage[parfill]{parskip}
\usepackage{lmodern}
\usepackage{tikz}
	\usepackage{physics}
%\documentclass[tikz, border=2mm]{standalone}
\usepackage{karnaugh-map}
%\documentclass{article}
\usepackage{tabularx}
\usepackage{circuitikz}
\usetikzlibrary{calc}
\usepackage{amsmath}
\usepackage{amssymb}
\renewcommand*\familydefault{\sfdefault}
\usepackage{watermark}
\usepackage{lipsum}
\usepackage{xcolor}
\usepackage{listings}
\usepackage{float}
\usepackage{titlesec}
\providecommand{\mtx}[1]{\mathbf{#1}}
\titlespacing{\subsection}{1pt}{\parskip}{3pt}
\titlespacing{\subsubsection}{0pt}{\parskip}{-\parskip}
\titlespacing{\paragraph}{0pt}{\parskip}{\parskip}
\newcommand{\figuremacro}[5]

\begin{document}

\title{\mytitle}
\author{\myauthor\hspace{1em}\\\contact\\FWC22036\hspace{6.5em}IITH\hspace{0.5em}\mymodule\hspace{6em}ASSIGN-4}
\date{}
	\maketitle
		\begin{figure}
	\tableofcontents
\vspace{10mm}
   \section{Problem}

   Given $ AP\: || \: BQ\: || CR $  . Prove that\\ 
   ar(AQC)= ar(PBR)\\
  
       \centering
        \includegraphics[scale=1]{../../../Downloads/diag_1.png}  
        \label{fig:2}
    \end{figure}
   \section{Solution}
   \textbf{Theory:}\\
   Given $ AP\: || \: BQ\: || CR $ \\
\textbf{To Prove:} Ar(AQC)=Ar(PBR) \\
$\Delta$ BQA and $\Delta$ BQP lies on same base  and are between same parallel BQ and AP\\
$\therefore$ Ar($\Delta$ BQA)=Ar($\Delta$ BQP)......(1)\\ 
\textbf{Theorem} : Two triangles on the same base (or equal bases) and between the same parallels are equal in area.
\begin{center}
$\therefore$ Ar($\Delta$ BQC)=Ar($\Delta$ BQR)......(2)\\ 
\end{center}
\textbf{To Prove:}  Ar(AQC)=Ar(PBR)\\
Add (1) to (2) \\
Ar($\Delta$BQA)+Ar(BQC)=Ar($\Delta$BQP)+Ar(BQR)
\begin{center}
$\therefore$ Ar(AQC)=Ar(PBR)     \\
Hence, Proved \\
\
\\
\
\\
\end{center}
\vspace{3mm}
\textbf{Termux commands :}
\begin{lstlisting}
python3 matrixline.py
\end{lstlisting}


The input parameters for this construction are 
\begin{center}
\begin{tabular}{|c|c|c|}
	\hline
	\textbf{Symbol}&\textbf{Value}&\textbf{Description}\\
	\hline
	k&1&AB\\
	\hline
	c&8&CA\\
	\hline
	a&12&CR\\
	\hline
	p2&4&AP\\
	\hline
	${\theta}$& $\pi/3$&$ \angle $AC\\  
	\hline
	C&$\
	\begin{pmatrix}
		0 \\
		0 \\
	\end{pmatrix}$%
	&Point C\\
	
	\hline
\end{tabular}
\end{center}
\textbf{To Prove:} Ar(AQC)=Ar(PBR)
  %	\begin{align}
	%		\vec{C} &= \myvec{0 \\ 0}, \vec{E}=\myvec{-5 \\ 3}\\
	%			\vec{F} &= \myvec{3 \\ 0}, \vec{D}=\myvec{-3 \\ 0}
	%	\end{align}
		\begin{center}
	v1=A-C\\
	v2=A-Q\\
	\vspace{3mm}
	Area of the triangle $\Delta$AQC is given by \\
Ar($\Delta$AQC) =$\frac{1}{2}$$\norm{\vec{v1}\times\vec{v2}}$............(2)\\
\vspace{3mm}
v3=R-P\\
	v4=R-B\\
	\vspace{3mm}
		Area of the triangle $\Delta$PBR is given by \\
 Ar($\Delta$PBR) =$\frac{1}{2}$$\norm{\vec{v3}\times\vec{v4}}$...............(3)
	\end{center}
	\begin{center}
$\therefore$ Ar(AQC)=Ar(PBR)\\

\end{center}
\vspace{10mm}
The below python code realizes the above construction:	\\

\url{https://github.com/KrishnaYadati/Assignments/tree/main/Matrix-line_ assignment/line_program}
 \section{Construction}
 	\begin{center}
    \includegraphics[scale=1]{../../python/figs/matrix.pdf} 
     Figure of Construction
  	\end{center}
  	  
\bibliographystyle{ieeetr}
\end{document}

