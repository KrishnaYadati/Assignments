
\documentclass{article}

\usepackage[english]{babel}

\usepackage[letterpaper,top=2cm,bottom=2cm,left=3cm,right=3cm,marginparwidth=1.75cm]{geometry}

% Useful packages
\usepackage{multicol}
\usepackage{karnaugh-map}
\usepackage{amsmath}
\usepackage{graphicx}
\usepackage{array}
\usepackage{blindtext}
\usepackage[utf8]{inputenc}

\usepackage[colorlinks=true, allcolors=blue]{hyperref}
\title{K-MAP for POS Expression}
\author{YADATI KRISHNA}

\begin{document}
\maketitle
\begin{multicols}{2}
\tableofcontents

\begin{abstract}
 This manual shows how to use Arduino and LED to represent the K-MAP for POS expression for the function "G" shown in below truth table. \\
 
 \centering
 
 \begin{tabular}{ |c |c |c |c |}
 \hline
 U  &  V  &  W  &  G\\
 \hline
 0  &  0  &  0  &  1\\
 \hline
 0  &  0  &  1  &  0\\
 \hline
 0  &  1  &  0  &  1\\
 \hline
 0  &  1  &  1  &  0\\
 \hline
 1  &  0  &  0  &  1\\
 \hline
 1  &  0  &  1  &  0\\
 \hline
 1  &  1  &  0  &  0\\
 \hline
 1  &  1  &  1  &  1\\
 \hline
 \end{tabular}
 
\end{abstract}
\section{Components}

%\begin{table}[]
    \centering
    \begin{tabular}{ |c |c |c |c |}
\hline
\textbf{Components} & \textbf{Value} & \textbf{Quantity} \\
\hline
  
 Arduino & UNO & 1 \\  
 
 
 LED & - & 1 \\
 Jumper wires&M-M &3\\
 Breadboard& &1\\
 \hline
 \end{tabular}
 \vspace{3mm}
 
 %\caption{Table 1.0}
    \label{table1}
%\end{table}

\section{Hardware}

\textbf{Problem 2.1} Make connections between the Arduino and LED using Breadboard.
\section{Software}

\textbf{Problem 3.1.} execute the following program after
downloading.
\framebox{
\url{https://github.com/KrishnaYadati/Assignments/assignment3}}
\vspace{3mm}
%\begin{table}[]

 \vspace{3mm}
 
 %\caption{Table 1.0}
    \label{table1}
%\end{table}
\begin{karnaugh-map}[4][2][1][$VW$][$U$]
        \minterms{0,2,4,7}
        \maxterms{1,3,5,6}

        \implicant{1}{5}
        \implicant{1}{3}
       \implicant{6}{6}
    \end{karnaugh-map}
    \centering
    \textbf{K-MAP}
    \vspace{3mm}
    
%\end{flushleft}

U, V, W are the
inputs and LED is the output.Using boolean logic.
\begin{equation}
G=(V+W')(U+W')(U'+V'+W) 
\end{equation}

\end{multicols}{}
\end{document}
