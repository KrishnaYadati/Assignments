\documentclass{article}

% Language setting
% Replace `english' with e.g. `spanish' to change the document language
\usepackage[english]{babel}

% Set page size and margins
% Replace `letterpaper' with `a4paper' for UK/EU standard size
\usepackage[letterpaper,top=2cm,bottom=2cm,left=3cm,right=3cm,marginparwidth=1.75cm]{geometry}

% Useful packages
\usepackage{multicol}
\usepackage{amsmath}
\usepackage{graphicx}
\usepackage{array}
\usepackage{blindtext}
\usepackage{hyperref}
\usepackage[utf8]{inputenc}

\usepackage[colorlinks=true, allcolors=blue]{hyperref}

\title{POS Expression for Paper 2013 Question 6)c}
\author{YADATI KRISHNA}

\begin{document}
\maketitle
\begin{multicols}{2}
\tableofcontents
\vspace{10mm}

\begin{abstract}
 This manual shows how to use Arduino with LED to represent POS expression.
\end{abstract}
\section{Components}

%\begin{table}[]
    \centering
    \begin{tabular}{ |c |c |c |c |}
\hline
\textbf{Components} & \textbf{Value} & \textbf{Quantity} \\
\hline
 %Resistor & 220Ohm & 1 \\ 
 Arduino & UNO & 1 \\  
 %Seven segment Display &  & 1 \\
 %Decoder& 7447&1 \\
 LED & - & 1 \\
 Jumper wires&M-M &3\\
 Breadboard& &1\\
 \hline
 \end{tabular}
 \vspace{3mm}
 
 %\caption{Table 1.0}
    \label{table1}
%\end{table}

\section{Hardware}

\textbf{Problem 2.1} Make connections between the Arduino and LED using Breadboard.

%\begin{figure}
 %    \centering
 %    \includegraphics{7447_pin.png}
  %   \includegraphics{7-Segment-Display-Pinout.jpg}
%\caption{7447 pin diagram}
 %    \label{fig:7447}
  %   \textbf{Problem 2.2}  Make connections to the lower pins of
%the 7447 according to Table 2.2 and connect VCC =
%5V. 
%\hfill
%\vspace{10mm}
%\includegraphics{7-Segment-Display-Pinout.jpg}
  %   \includegraphics{7-Segment-Display-Pinout.jpg}
%\caption{seven segment diagram}
 %    \label{fig:seven}
\section{Software}

\textbf{Problem 3.1.} Now make the connections as per
Table 3.1 and execute the following program after
downloading.

\vspace{10mm}
\framebox{
\url{https://github.com/KrishnaYadati/Assignments}}

\vspace{10mm}
%\begin{table}[]
    \centering
    \begin{tabular}{ |c |c |c |c| c|}

%\textbf{Components} & \textbf{Value} & \textbf{Quantity} \\
\hline
 \textbf{}  & W& V & U \\ 
 \hline
 \textbf{Input}  & 0 & 0 & 1\\
 \textbf{Arduino}  & 8 & 7 & 6 \\  
 \hline
 \end{tabular}
 \vspace{3mm}
 
 %\caption{Table 1.0}
    \label{table1}
%\end{table}


U, V, W are the
inputs and LED is the output.Using boolean
logic,
\begin{equation}
G= (U+V+W') (U+V'+W') (U'+V+W') (U'+V'+W)
\end{equation}
%D = (U || V') && W || Z
\end{multicols}{}
\end{document}
